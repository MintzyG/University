
% UENF - CCT - LCMAT - Curso de Ci\^{e}ncia da Computa\c{c}\~{a}o
% Campos, RJ,  2023
% Disciplina: Paradigmas de Linguagens de Programa\c{c}\~{a}o
% Aluno: Eric Hoffmann Fernandes Braga

\chapterimage{ScalaH} % Chapter heading image ==>  Trocar este arquivo por outro 1200x468
\chapter{ Conceitos b\'{a}sicos da Linguagem Rust}

Neste cap\'{i}tulo ser\'{a} apresentado os conceitos basicos da linguagem Rust como vistos nos livros \cite{Klab14}, \cite{Car15} e \cite{Car16}. Estes livros sao os livros basicos para o estudo linguagem Rust.

%%%%%%%%=================================
\section{Vari\'{a}veis e constantes}
%%%%%%%%=================================

Diferente de muitas linguagens em Rust o padrão para variaveis e ser imutavel/constante, isso significa que uma variavel não pode receber outro valor depois que inicialmente atribuida, essa decisão feita pela fundacão Rust garante que voce possa aproveitar da segurança e concorrência que Rust oferece mais facilmente, mas você ainda tem a opcão de transformar uma variavel em mutavel caso deseje.
\par
\begin{lstlisting}[language=rust]
fn main() {
  // Cria uma variavel imutavel
  let a = 2; 

  // Cria uma variavel mutavel
  let mut b = 3 
}
\end{lstlisting}
\pagebreak
\newpage
Vamos ver exemplos de como isso difere de uma linguagem como C:
\begin{lstlisting}[language=rust]
// variables.rs
fn main() {
  // Cria uma variavel imutavel
  let x = 2; 

  // O esperado aqui em C e que se escreva o numero 2 no console
  println!("O valor de x e: {x}");
  
  x = 3;

  // O esperado aqui em C e que se escreva o numero 3 no console
  println!("O valor de x e: {x}");
}
\end{lstlisting}
\begin{lstlisting}[language=bash]
$ rustc variables.rs
$ ./variables
> 2
> 2
\end{lstlisting}
\par
Aqui podemos ver que apesar de termos feito x = 3 no código a variavel continuou com o valor de '2' isso acontece pois as variaves sao imutaveis por padrão, logo, se quisermos mudar o valor de uma variavel em Rust temos que explicitamente declará-la como mutavel usando 'mut'.
\begin{lstlisting}[language=rust]
// variables.rs
fn main() {
  // Cria uma variavel imutavel
  let mut x = 2; 

  println!("O valor de x e: {x}");
  
  x = 3;

  println!("O valor de x e: {x}");
}
\end{lstlisting}
\begin{lstlisting}[language=bash]
$ rustc variables.rs
$ ./variables
> 2
> 3
\end{lstlisting}
\par
Agora podemos ver que nosso código esta funcionando como esperavamos, e o valor de x muda quando nos atribuimos o valor 3 a ele no codigo, isso ocorre pois declaramos a variavel como 'mut'.
\pagebreak
\newpage

\subsection{Constantes}

Como variaveis imutaveis, constantes são valores que se ligam a um nome e não podem mudar, mas existem algumas diferenças entre constantes e variaveis. Primeiro, voce não pode usar 'mut' em uma constante, pois elas nao são só imutaveis por padrão, elas são imutaveis sempre. Segundo, constantes se declaram com a palavra chave 'const' ao inves de 'let' e o seu tipo deve obrigatoriamente ser definido usando \textit{type-annotations}, não se preocupe, logo em seguida voce descobrirá o que são \textit{type annotations}.
\par
Constantes podem ser declaradas em qualquer escopo, incluindo o global, e a última diferença é que constantes só podem ser atribuidas valores de expressões constantes, significando que só não podem ser valores que so sao possíveis de se computar na execução do programa.
\par
Aqui está um exemplo da declaração de uma constante:

\begin{lstlisting}[language=rust]
// constant.rs
fn main() {
  // Cria uma constante
  const MyConst: u32 = 2; 
}
\end{lstlisting}

\subsection{Shadowing}

\textit{Shadowing} é o nome que programadores de Rust deram pra quando você re-declara uma váriavel que já existia antes, assim você sobrepõe o valor antigo da variavel com o valor novo, essa tecnica é comumente usada com váriaveis imutáveis. Uma váriavel que sofreu \textit{shadowing} continua existindo como seu novo valor até que ela mesma sofra \textit{shadowing} ou que acabe o escopo em que ela exista, isso permite padrões avancados de código como:

\begin{lstlisting}[language=rust]
// shadowing.rs
fn main() {
  // Cria uma variavel imutavel x
  let x: u32 = 5;

  // Aplica o shadowing dentro deste escopo
  // transformando a variavel em 6
  let x = x + 1;
  {
    // Aplica shadowing dentro deste escopo
    // transformando a variavel em 12
    let x = x * 2
    println!("Valor de X no escopo interior e: {x}");
  }
  // O escopo finaliza logo a variavel volta a ser 6
  println!("Valor de X no escopo exterior e: {x}");
}
\end{lstlisting}

\begin{lstlisting}[language=bash]
$ rustc shadowing.rs
$ ./shadowing
> Valor de X no escopo interior e: 12
> Valor de X no escopo exterior e: 6
\end{lstlisting}

\textit{Shadowing} se difere de declarar uma váriavel como 'mut' pois teremos um erro de compilação se tentarmos re-atribuir a váriavel sem usar a palavra chave 'let'. Usando 'let' nós podemos realizar algumas transformações em um valor mas ter o valor imutável depois que estas transformações acabarem. Outra vantagem de \textit{shadowing} e, como estamos basicamente criando outra variavel quando usamos 'let', nós podemos mudar o valor da variavel, mas reusar o mesmo nome.

%%%%%%%%=================================
\section{Tipos de Dados B\'{a}sicos}
%%%%%%%%=================================

Assim como outras linguagens Rust possui diversos tipos de dados que permitem que o compilador ache erros no código e otimize onde possível, mas a linguagem também possui tipos dinâmicos permitindo que os tipos que devem ser usados sejam inferidos pelo compilador durante a etapa de compilação sem que o usuário especifique explicitamente.
\par
Então se apenas declararmos a váriavel, o compilador e encarregado de decidir o tipo apropriado da váriavel em questão, porem se quisermos ter certeza de que estamos trabalhando com um tipo de dado específico podemos definí-lo usando \textit{type-annotations} onde entre a váriavel e seu valor decidimos seu tipo.
\begin{lstlisting}[language=rust]
fn main() {
  // O compilador infere o tipo deste dado 
  let a = 2; 

  // Nos informamos o tipo u32 (Inteiro de 32 bits sem sinal)
  let b: u32 = 3 
}
\end{lstlisting}
\subsection{Inteiros}
Em Rust se quisermos trabalhar com interios temos opcoes desde i8 ate i128 onde o número decide quantos bits de informação aquela váriavel pode guardar, então, um i8 pode guardar desde os numeros -127 até o número 128. Se não quisermos números negativos podemos especificar trocando a letra de 'i' para 'u' que significa \textit{unsigned} (sem sinal) assim um u8 pode guardar desde 0 ate 255 e ainda temos um tipo especial 'isize' que se define de acordo com a arquitetura de seu computador, sendo um i32 se o computador for 32 bits ou um i64 se o computador for 64 bits. Aqui estão todos os inteiros:
\begin{lstlisting}[language=rust]
//types.rs
fn main() {
  // O compilador infere o tipo deste dado 
  let a = 0; 

  let b: i8 = 1;
  let c: i16 = -2;
  let d: i32 = 3;
  let e: i64 = -4;
  let f: i128 = 5;
  let g: isize = -6;
  ...
\end{lstlisting}
\newpage
\begin{lstlisting}[language=rust]
//types.rs
  ...
  let A: u8 = 1;
  let B: u16 = 2;
  let C: u32 = 3;
  let D: u64 = 4;
  let E: u128 = 5;
  let F: usize = 6;
  ...
\end{lstlisting}
\par
Inteiros tambem pode ser escritos em outras bases numerais que nao sejam base decimal
\par
\begin{table}[h]
\centering
\begin{tabular}{|l|l|}
\hline
Literais & Exemplo      \\ \hline
Decimal  & 1\_580\_000  \\ \hline
Hex      & 0xff         \\ \hline
Octal    & 0o77         \\ \hline
Binário  & 0b1111\_0000 \\ \hline
Byte     & b'a'         \\ \hline
\end{tabular}
\end{table}

\begin{lstlisting}[language=rust]
//types.rs
  ...
  // 'Underlines' podem ser usados para melhor legibilidade
  let Decimal = 1_580_00;
  let Hex = 0xff;
  let Octal = 0o77;
  let Binario = 0b1111_0000;
  let Byte = b'a';
  ...
\end{lstlisting}

%%%%%%%%=================================
\subsection{Floats}
%%%%%%%%=================================

Rust possui dois tipos primitivos para \textit{floats}, os quais são números com casas decimais.
Os tipos de \textit{float} do Rust sao 'f32' e 'f64 que são de tamanhos 32 bit e 64 bit respectivamente. O valor padrão de um \textit{float} e 'f64' pois são capazes de mais precisão. Todos os  \textit{floats} possuem sinal.

\begin{lstlisting}[language=rust]
//types.rs
  ...
  let fa = 1.2; // f64

  let fb: f32 = 3.1415 // f32
  ...
\end{lstlisting}

\textit{Floats} em Rust seguem o padrao IEEE-754. O tipo 'f32' é de precisão única e o tipo 'f64' é de precisão dupla.
\pagebreak
\newpage

%%%%%%%%=================================
\subsection{Booleano}
%%%%%%%%=================================

O tipo booleano em Rust como em qualquer outra linguagem tem dois possiveis valores, \textit{true} e \textit{false} e são criadas usando a palavra chave 'bool'. Como qualquer outra tipo ele pode ser inferido pelo compilador ou ser \textit{type-annotated} pelo programador

\begin{lstlisting}[language=rust]
//types.rs
  ...
  let boolTrue = true; 

  let boolFalse: bool = false // type-annotated
  ...
\end{lstlisting}

O uso de booleans veremos mais tarde na seção de controle de fluxo

%%%%%%%%=================================
\subsection{Caracteres}
%%%%%%%%=================================

\begin{lstlisting}[language=rust]
//types.rs
  ...
  let w = 'w';
  let z: char = 'z' \\ type-annotated
}
\end{lstlisting}

Cáracteres são o tipo alfabético mais simples da linguagem, diferentemente de strings cáracteres são envolvidos em aspas simples ao invés de aspas duplas. O tipo char tem 4 bytes em tamanho e representa valores escalares de Unicode, o que significa que os caracteres no Rust podem representar desde caracteres ASCII, à caracteres Chineses à até emojis.

%%%%%%%%=================================
\subsection{Strings}
%%%%%%%%=================================

A fazer:
\begin{itemize}
  \item String
  \item \&str
  \item \&'static str
  \item Box str
  \item Rc str
  \item Arc str
  \item byte str
  \item String literals
  \item Specialized String
  \item Interoperable String
\end{itemize}

%%%%%%%%=================================
\section{Tipos de Dados Compostos}
%%%%%%%%=================================

%%%%%%%%=================================
\subsection{Tuplas}
%%%%%%%%=================================

As tuplas são uma maneira de agrupar multiplos valores de diferentes tipos em um unico tipo composto, tuplas possuem tamanho fixo e não podem diminuir ou aumentar depois de criadas. Nós escrevemos uma tupla criando uma lista separada por vírgulas, tuplas podem ser tanto inferidas quanto \textit{type-annotated}.
\pagebreak
\newpage
\begin{lstlisting}[language=rust]
//compounds.rs
fn main() {
  let tupa = (500, 6.4, 1);

  // Type-annotated
  let tupb: (i32, f64, u8) = (500, 6.4, 1);
  ...
\end{lstlisting}

Nós então com uma tupla podemos usar casamento de padrão para desconstruir a tupla e atribuir cada elemento a uma váriavel.

\begin{lstlisting}[language=rust]
//compounds.rs
  ...
  let tup = (500, 6.4, 1);
  let (x, y, z) = tup;
  println!("O valor de x e {x} e z {z}")
  ...
\end{lstlisting}

Também podemos acessar um elemento da tupla diretamente usando a notacão de ponto:

\begin{lstlisting}[language=rust]
//compounds.rs
  ...
  let quinhentos = tup.0;
  let seis_ponto_quatro = tup.1;
  let um = tup.2;
  ...
\end{lstlisting}

\begin{lstlisting}[language=bash]
$ rustc compounds.rs
$ ./compounds
> Valor de X e 500 e z 1
\end{lstlisting}

Uma tupla sem nenhum valor tem um nome especial, \textit{unit}. Esse valor e seu tipo ambos sao escritos '()' e representam um valor vazio ou tipo de retorno vazio. Expressões implicitamente retornam o valor \textit{unit} se elas não retornam nenhum outro valor.

\pagebreak
\newpage

%%%%%%%%=================================
\subsection{Arrays}
%%%%%%%%=================================

Outra maneira de ter uma coleção de valores e um \textit{array}, porém diferente de uma tupla, todos os elementos de um \textit{array} devem ser do mesmo tipo. Diferente de outras linguagens \textit{arrays} em Rust tem tamanho fixo. Valores em um array são escritos em uma lista separada por vírgulas dentro de colchetes.

\begin{lstlisting}[language=rust]
//compounds.rs
  ...
  // O tamanho e o tipo sao inferidos 
  let arr = [1, 2, 3, 4, 5];
  ...
\end{lstlisting}

\textit{Arrays} são úteis quando você sabe que o número de elementos não muda. Por exemplo um \textit{array} com os 12 meses do ano. Um array também como outras váriaveis pode ser inicializado usando \textit{type-annotations}.

\begin{lstlisting}[language=rust]
//compounds.rs
  ...
  // O tamanho e o tipo sao descritos
  let arrAnnotated: [i32, 5] = [1, 2, 3, 4, 5];

  // Tambem e possivel inicializar um array com o mesmo valor
  // em todas as suas posicoes
  let same = [3; 5];
  // Resulta em = [3, 3, 3, 3, 3]
\end{lstlisting}

Podemos acessar os elemenos de um array usando colchetes e a posição após seu nome

\begin{lstlisting}[language=rust]
//compounds.rs
  ...
  let arr = [1, 2, 3, 4, 5];
  
  let first = a[0];
  let middle = a[2]
  let last = a[4];
}
\end{lstlisting}

Se um elemento fora do array tentar ser acessado no código, a \textit{engine} do Rust avaliará a expressão para um pânico e terminara a execução do código, resultando no seguinte resultado no terminal:

\begin{lstlisting}[language=bash]
thread 'main' panicked at 'index out of bounds: 
the len is 5 but the index is 10', src/main.rs:19:19
note: run with `RUST_BACKTRACE=1` 
environment variable to display a backtrace
\end{lstlisting}

\pagebreak
\newpage

%%%%%%%%=================================
\section{Operadores e Express\~{o}es em Rust}
%%%%%%%%=================================

%%%%%%%%=================================
\subsection{Operadores}
%%%%%%%%=================================

Operadores disponíveis em Rust se assimilam com os operadores disponíveis em C com algumas excessões específicas que veremos agora.

\begin{lstlisting}[language=rust]
//operators.rs
fn main() {
  // Operacoes matematicas
  1 + 2; // Soma
  2 - 1; // Subtracao
  2 * 4; // Multiplicacao
  4 / 2; // Divisao

  true && false; // Operacao logica AND
  true || false; // Operacao logica OR
  !true; // Operacao logica NOT

  0b1111 & 0b11111; // Operacao Bitwise AND
  0b0110 | 0b0101; // Operacao Bitwise OR
  0b0110 ^ 0b1001; // Opercao Bitwise XOR

  1u32 << 5; // Opercao Bitshift Left
  32u32 >> 5; // Operacao Bitshift Right

  // Excessao nao presente em linguagens tipo C
  -2.5e-7; // Notacao cientifica
}
\end{lstlisting}

%%%%%%%%=================================
\subsection{Express\~{o}es}
%%%%%%%%=================================

Um programa em Rust é majoritariamente composto de declarações

\begin{lstlisting}[language=rust]
//expressions.rs
fn main() {
  // declaracao 
  // declaracao 
  // declaracao 
  // declaracao 
}
\end{lstlisting}

\pagebreak
\newpage

Existem alguns tipos de declarações em Rust, as duas mais comums são a atribuição de uma váriavel e usar ';' com uma expressão.

\begin{lstlisting}[language=rust]
//expressions.rs
...
    // atribuicao de variavel
    let x = 5;

    // expressoes;
    x;
    x + 1;
    15;
...
\end{lstlisting}
Blocos tambem sao expressões, logo eles podem ser usados como valores em atribuições
\begin{lstlisting}[language=rust]
//expressions.rs
...
  let x = 5u32;
  let y = {
    let x_quadrado = x * x;
    let x_cubo = x_squared * x;

    // esta expressao vai ser atribuida a `y`
    x_cube + x_squared + x // a falta do ';' define o retorno
  };

  // O ';' suprime a expressao dentro do {}
  let z = {
    // logo a expressao {} nao tem valor, retornando '()'
    2 * x;
  };

  println!("x e {:?}, y e {:?} e z e {:?}", x, y, z);
}
\end{lstlisting}
\begin{lstlisting}[language=bash]
$ rustc expressions.rs
$ ./expressions
> x e 5, y e 155 e z e ()
\end{lstlisting}
